%%%%%%%%%%%%%%%%%%%%%%%%%%%%%%%%%%%%%%%%%
% Arsclassica Article
% LaTeX Template
% Version 1.1 (1/8/17)
%
% This template has been downloaded from:
% http://www.LaTeXTemplates.com
%
% Original author:
% Lorenzo Pantieri (http://www.lorenzopantieri.net) with extensive modifications by:
% Vel (vel@latextemplates.com)
%
% License:
% CC BY-NC-SA 3.0 (http://creativecommons.org/licenses/by-nc-sa/3.0/)
%
%%%%%%%%%%%%%%%%%%%%%%%%%%%%%%%%%%%%%%%%%

%----------------------------------------------------------------------------------------
%	PACKAGES AND OTHER DOCUMENT CONFIGURATIONS
%----------------------------------------------------------------------------------------

\documentclass[
10pt, % Main document font size
a4paper, % Paper type, use 'letterpaper' for US Letter paper
oneside, % One page layout (no page indentation)
%twoside, % Two page layout (page indentation for binding and different headers)
headinclude,footinclude, % Extra spacing for the header and footer
BCOR5mm, % Binding correction
]{scrartcl}


\usepackage[
nochapters,
beramono, 
eulermath
pdfspacing, 
dottedtoc 
]{classicthesis} 

\usepackage{arsclassica} 

\usepackage[T1]{fontenc} 

\usepackage[utf8]{inputenc} 

\usepackage{graphicx} 
\graphicspath{{Figures/}} 

\usepackage{enumitem} 

\usepackage{lipsum} 

\usepackage{subfig}

\usepackage{amsmath,amssymb,amsthm} 

\usepackage{varioref} 
\usepackage{siunitx}
\usepackage{makeidx}
\usepackage{minted}
\usepackage{tabularx}
\usepackage{longtable}
\usepackage{lscape}
\makeindex
\usepackage{glossaries}
\makeglossaries


\theoremstyle{definition} 
\newtheorem{definition}{Definition}

\theoremstyle{plain} 
\newtheorem{theorem}{Theorem}

\theoremstyle{remark}

\hypersetup{
%draft, % Uncomment to remove all links (useful for printing in black and white)
colorlinks=true, breaklinks=true, bookmarks=true,bookmarksnumbered,
urlcolor=webbrown, linkcolor=RoyalBlue, citecolor=webgreen, % Link colors
pdftitle={}, % PDF title
pdfauthor={\textcopyright}, % PDF Author
pdfsubject={}, % PDF Subject
pdfkeywords={}, % PDF Keywords
pdfcreator={pdfLaTeX}, % PDF Creator
pdfproducer={LaTeX with hyperref and ClassicThesis} % PDF producer
}


 % Include the structure.tex file which specified the document structure and layout
\newglossaryentry{SA}{
name={SOTA},
description={State of the art. Défini les standards les plus exigents de l’industrie et notamment pour la recherche, dans le domaine de la vision par ordinateur}
}

\newglossaryentry{DLV3}{
name={DeepLabV3},
description={Algorithme de deep learning utilisant le principe d’encodeur et décodeur avec des convolutions séparables Atrous pour la segmentation sémantique d’image}
}

\newglossaryentry{FEATS}{
name={features},
description={Singularités dans l’image, capturées à travers des filtres et à des échelles multiples, puis pouvant être mises en correspondance entre les niveaux pour obtenir des prédictions de classes d’appartenance pour chaque pixel de l’image. Chaque singularités capturées pouvant servir d’entrée dans une autre sous-partie du réseau de neurones à des fins de délimitation de frontière d’objets plus ou moins diffus de l’image comme par exemple le ciel ou un véhicule}
}

\newglossaryentry{FCN}{
name={FCN},
description={Ce sont des réseaux convolutifs ou "convnets", qui sont complètement connectés, i.e. chaque neurone de sortie est connecté aux neurones d’entrée. Ils permettent non seulement la classification de l’image entière, mais en en ayant permis des progrès sur les tâches locales, pourvoyant des outputs structurés}
}


\newglossaryentry{MIOU}{
name={mIoU},
description={La moyenne d’une « Intersection over Union » (IoU), aussi connue sous le nom de l’index Jaccard, est la métrique d’évaluation la plus populaire pour la tâche de segmentation, de la détection d’objet ainsi que du suivi}
}

\newglossaryentry{CFM}{
name={CFM},
description={C’est une méthode (et même un outil), qui relie des extractions de features issues de la convolution puis ensuite, extrait des segments depuis des cartes de features plutôt qu’à partir d’une image brute. Ces segments donnés par R-CNN, sont projetés sur la dernière couche. Les segments projetés jouent un rôle de fonction binaire : les features masquées sont transmises à la couche complètement connectée (fully-connected) pour la reconnaissance}
}


\hyphenation{Fortran hy-phen-ation} % Specify custom hyphenation points in words with dashes where you would like hyphenation to occur, or alternatively, don't put any dashes in a word to stop hyphenation altogether

%----------------------------------------------------------------------------------------
%	TITLE AND AUTHOR(S)
%----------------------------------------------------------------------------------------

\title{\normalfont\spacedallcaps{Segmentation sémantique d’image}} % The article title

%\subtitle{Subtitle} % Uncomment to display a subtitle

\author{\spacedlowsmallcaps{Romain BOYRIE*}}%\textsuperscript{1}}} % The article author(s) - author affiliations need to be specified in the AUTHOR AFFILIATIONS block

\date{$21$ Janvier $2022$} % An optional date to appear under the author(s)

%----------------------------------------------------------------------------------------

\begin{document}

%----------------------------------------------------------------------------------------
%	HEADERS
%----------------------------------------------------------------------------------------

\renewcommand{\sectionmark}[1]{\markright{\spacedlowsmallcaps{#1}}} % The header for all pages (oneside) or for even pages (twoside)
%\renewcommand{\subsectionmark}[1]{\markright{\thesubsection~#1}} % Uncomment when using the twoside option - this modifies the header on odd pages
\lehead{\mbox{\llap{\small\thepage\kern1em\color{halfgray} \vline}\color{halfgray}\hspace{0.5em}\rightmark\hfil}} % The header style

\pagestyle{scrheadings} % Enable the headers specified in this block

%----------------------------------------------------------------------------------------
%	TABLE OF CONTENTS & LISTS OF FIGURES AND TABLES
%----------------------------------------------------------------------------------------

\maketitle % Print the title/author/date block

\setcounter{tocdepth}{2} % Set the depth of the table of contents to show sections and subsections only

\tableofcontents % Print the table of contents

\listoffigures % Print the list of figures

\listoftables % Print the list of tables

%----------------------------------------------------------------------------------------
%	ABSTRACT
%----------------------------------------------------------------------------------------

\section*{Abstract} % This section will not appear in the table of contents due to the star (\section*)

Ce document détaille l’étendue des prolégomènes du domaine réservé à la segmentation sémantique d’image sur une architecture basée en réseau neuronal profond. À cet effet, nous avons résumé des papiers scientifiques, en profitant de la liberté de circulation de ses papiers sur des plateformes, tel Arxiv. Il nous appartient également d’établir une présentation de notre modèle en exposant nos résultats. Dans un premier temps, ce document établit l’état de l’art (\gls{SA}) en contextualisant la recherche sur les dix dernières années seulement. Nous présentons les concepts qui ont surpassés les performances de SOTA\index{SOTA} en leur temps, en tant que production scientifique, dans toute l’acception de ce mot, tant au niveau des outils que des concepts découverts. Nous conserverons un droit de réserve sur notre exhaustivité mais nous partagerons le code source. Dans une seconde partie, nous présentons l’architecture qui se démarque, \index{DeepLabV3}\gls{DLV3}, et qui a permis de réaliser $82.1\%$ \gls{MIOU} (mean of intersection over union) sur la base Cityscapes avec un apprentissage exempt de post traitement.

%----------------------------------------------------------------------------------------
%	AUTHOR AFFILIATIONS
%----------------------------------------------------------------------------------------

\let\thefootnote\relax\footnotetext{* \textit{Parcours en Intelligence Artificielle, Formation Open Classrooms, Paris, France}}

%\let\thefootnote\relax\footnotetext{\textsuperscript{1} \textit{Department of Chemistry, University of Examples, London, United Kingdom}}

%----------------------------------------------------------------------------------------

\newpage % Start the article content on the second page, remove this if you have a longer abstract that goes onto the second page

%----------------------------------------------------------------------------------------
%	INTRODUCTION
%----------------------------------------------------------------------------------------

\section{Introduction}

\paragraph{}Dès l’année $2014$, dans le domaine de la vision par ordinateur, les réseaux convolutifs complètement connectés (\index{FCN}\gls{FCN}) réalisent déjà des tâches de segmentation en décelant des hiérarchies de fonctionnalités que nous appellerons \index{features}\gls{FEATS}, qui sont des singularités appartenant à l’image dans ses $3$ dimensions. Les chercheurs de Microsoft explorent les possibilités offertes par le masquage des cartes de fonctionnalités de convolution, ou convolutional feature maps (\gls{CFM}) \cite{dai2015convolutional}.\paragraph{}\cite{cho2017xception}
 
%----------------------------------------------------------------------------------------
%	METHODS
%----------------------------------------------------------------------------------------

\section{Methods}

\lipsum[5] % Dummy text

\begin{enumerate}[noitemsep] % [noitemsep] removes whitespace between the items for a compact look
\item First item in a list
\item Second item in a list
\item Third item in a list
\end{enumerate}

%------------------------------------------------

\subsection{Paragraphs}

\lipsum[6] % Dummy text

\paragraph{Paragraph Description} \lipsum[7] % Dummy text

\paragraph{Different Paragraph Description} \lipsum[8] % Dummy text

%------------------------------------------------

\subsection{Math}

\lipsum[4] % Dummy text

\begin{equation}
\cos^3 \theta =\frac{1}{4}\cos\theta+\frac{3}{4}\cos 3\theta
\label{eq:refname2}
\end{equation}

\lipsum[5] % Dummy text

\begin{definition}[Gauss] 
To a mathematician it is obvious that
$\int_{-\infty}^{+\infty}
e^{-x^2}\,dx=\sqrt{\pi}$. 
\end{definition} 

\begin{theorem}[Pythagoras]
The square of the hypotenuse (the side opposite the right angle) is equal to the sum of the squares of the other two sides.
\end{theorem}

\begin{proof} 
We have that $\log(1)^2 = 2\log(1)$.
But we also have that $\log(-1)^2=\log(1)=0$.
Then $2\log(-1)=0$, from which the proof.
\end{proof}

%----------------------------------------------------------------------------------------
%	RESULTS AND DISCUSSION
%----------------------------------------------------------------------------------------

\section{Results and Discussion}

Reference to Figure~\vref{fig:gallery}. % The \vref command specifies the location of the reference

\begin{figure}[tb]
\centering 
\includegraphics[width=0.5\columnwidth]{GalleriaStampe} 
\caption[An example of a floating figure]{An example of a floating figure (a reproduction from the \emph{Gallery of prints}, M.~Escher,\index{Escher, M.~C.} from \url{http://www.mcescher.com/}).} % The text in the square bracket is the caption for the list of figures while the text in the curly brackets is the figure caption
\label{fig:gallery} 
\end{figure}

\lipsum[10] % Dummy text

%------------------------------------------------

\subsection{Subsection}

\lipsum[11] % Dummy text

\subsubsection{Subsubsection}

\lipsum[12] % Dummy text

\begin{description}
\item[Word] Definition
\item[Concept] Explanation
\item[Idea] Text
\end{description}

\lipsum[12] % Dummy text

\begin{itemize}[noitemsep] % [noitemsep] removes whitespace between the items for a compact look
\item First item in a list
\item Second item in a list
\item Third item in a list
\end{itemize}

\subsubsection{Table}

\lipsum[13] % Dummy text

\begin{table}[hbt]
\caption{Table of Grades}
\centering
\begin{tabular}{llr}
\toprule
\multicolumn{2}{c}{Name} \\
\cmidrule(r){1-2}
First name & Last Name & Grade \\
\midrule
John & Doe & $7.5$ \\
Richard & Miles & $2$ \\
\bottomrule
\end{tabular}
\label{tab:label}
\end{table}

Reference to Table~\vref{tab:label}. % The \vref command specifies the location of the reference

%------------------------------------------------

\subsection{Figure Composed of Subfigures}

Reference the figure composed of multiple subfigures as Figure~\vref{fig:esempio}. Reference one of the subfigures as Figure~\vref{fig:ipsum}. % The \vref command specifies the location of the reference

\lipsum[15-18] % Dummy text

\begin{figure}[tb]
\centering
\subfloat[A city market.]{\includegraphics[width=.45\columnwidth]{Lorem}} \quad
\subfloat[Forest landscape.]{\includegraphics[width=.45\columnwidth]{Ipsum}\label{fig:ipsum}} \\
\subfloat[Mountain landscape.]{\includegraphics[width=.45\columnwidth]{Dolor}} \quad
\subfloat[A tile decoration.]{\includegraphics[width=.45\columnwidth]{Sit}}
\caption[A number of pictures.]{A number of pictures with no common theme.} % The text in the square bracket is the caption for the list of figures while the text in the curly brackets is the figure caption
\label{fig:esempio}
\end{figure}

\printindex
\clearpage
\printglossaries

%----------------------------------------------------------------------------------------
%	BIBLIOGRAPHY
%----------------------------------------------------------------------------------------

\renewcommand{\refname}{\spacedlowsmallcaps{References}} % For modifying the bibliography heading

\bibliographystyle{unsrt}

\bibliography{sample.bib} % The file containing the bibliography

%----------------------------------------------------------------------------------------

\end{document}